\documentclass[12pt,a4paper]{amsart}

\usepackage{amsmath}
\usepackage{amsfonts}
\usepackage{amssymb}
\usepackage{enumerate}
\usepackage{graphicx, epstopdf}

\title{Quotient Graphs}
\date{}

\newtheorem{Theorem}{Theorem}[section]
\newtheorem{Lemma}[Theorem]{Lemma}
\newtheorem{Corollary}[Theorem]{Corollary}

\theoremstyle{remark}
\newtheorem{Remark}[Theorem]{Remark}
\newtheorem{Observation}[Theorem]{Observation}

\theoremstyle{definition}
\newtheorem{Definition}[Theorem]{Definition}
\newtheorem{Example}[Theorem]{Example}

\newcommand{\adj}[1]{\sim_{\scriptscriptstyle #1}}
\newcommand{\nadj}[1]{\nsim_{\scriptscriptstyle #1}}
\DeclareMathOperator{\diam}{diam}
\DeclareMathOperator{\aut}{Aut}
\DeclareMathOperator{\orb}{Orb}
\DeclareMathOperator{\stab}{Stab}
\DeclareMathOperator{\vset}{V}
\DeclareMathOperator{\eset}{E}
\DeclareMathOperator{\dist}{d}
\newcommand{\Mod}[1]{\mathrm{\ (mod\ #1)}}

\begin{document}
\maketitle

\section{Introduction}
\label{sec:Intro}

The automorphism group of a graph induces a partition of the vertex set into orbits of its action on the graph. Intuitively, we think of the vertices in one orbit as being ``automorphically equivalent". It is natural, therefore, to define the \emph{quotient graph} with respect to the automorphism and study its relation to the original graph. Formally, if $\Gamma$ is a graph with automorphism group $G$, and $\mathcal \Pi$ is the partition of the vertex set $V(G)$ into orbits under the action of $G$, then the quotient graph $\Gamma/G$ has $\Pi$ as its vertex set, and any two distinct vertices $X$ and $Y$ of $\Gamma/G$ are adjacent whenever there are vertices $x$ and $y$ of $\Gamma$ such that $x \in X$ and $y \in Y$.

\begin{Definition}
If $\Gamma$ is a graph with automorphism group $\aut \Gamma = G$, the quotient graph of $G$ is the graph $\Gamma' = \Gamma/G$ having vertex set
\begin{equation*}
V(\Gamma') = \{\, X \mid X = \orb_G(x), \exists x \in V(\Gamma) \,\}
\end{equation*}
and edge set 
\begin{equation*}
E(\Gamma') = \{\, (X,Y) \mid X = \orb_G(x), Y = \orb_G(y), X \ne Y, \exists x, y \in V(G) \,\}.
\end{equation*}
\end{Definition}

\begin{Lemma}
\label{lem:xy=>x'y'}
If $X$ and $Y$ are two orbits of the automorphism group of a graph $\Gamma$, and a vertex $x \in X$ is adjacent to a vertex $y \in Y$, then every vertex of $X$ is adjacent to some vertex of $Y$.
\end{Lemma}
\begin{proof}
Let $x' \in X$. Since $X = \orb(x)$ is the orbit of $x$ under the automorphism group of $\Gamma$, there is some automorphism $\varphi$ of $\Gamma$ such that $\varphi(x) = x'$. Then, $\varphi(y) = y'$, $\exists y' \in Y = \orb(y)$. Since $x \adj{} y$, it follows that $x' \adj{} y'$, as required.
\end{proof}

\begin{Lemma}
\label{lem:PathCorr}
If $X_1, X_2, \ldots, X_k$ is path in the quotient graph $\Gamma' = \Gamma/G$ of a graph $G$, then there is a path $x_1, x_2, \ldots, x_k$ in $\Gamma$ with $x_i \in X_i$, $i = 1, 2, \ldots, k$.
\end{Lemma}
\begin{proof}
Since $X_1 \adj{\Gamma'} X_2$, there exist vertices $x_1 \in X_1$ and $x_2 \in X_2$ such that $x_1 \adj{\Gamma} x_2$. Similarly, there exist vertices $x'_2 \in X_2$ and $x'_3 \in X_3$ such that $x'_2 \adj{\Gamma} x'_3$. Then by Lemma~\ref{lem:xy=>x'y'} $x_2 \in X_2$ is adjacent to some vertex $x_3 \in X_3$. Thus, we have a sequence $x_1, x_2, x_3$ with $x_i \in X_i$, $i = 1, 2, 3$. Proceeding similarly, we obtain a sequence of vertices $x_1, x_2, \ldots, x_k$ with $x_i \in X_i$, $i = 1, 2, \ldots, k$. Observe that since each $X_i$ in the $\Gamma'$-path is distinct from every $X_j$, $j \ne i$, each $x_i$ in the sequence must be distinct from each $x_j$, $j \ne i$, which makes the sequence a path of the required form in $\Gamma$.
\end{proof}

\begin{Theorem}
\label{thm:CycCorr}
If $X_1, X_2, \ldots, X_k$ is a cycle in the quotient graph $\Gamma' = \Gamma/G$ of a graph $G$, then the vertices $X_1 \cup \cdots \cup X_k$ of $\Gamma$ contain an induced cycle.
\end{Theorem}
\begin{proof}
Let $X_1, X_2, \ldots, X_k$ be a cycle of length $k$ in the quotient graph $\Gamma'$. Then by Lemma~\ref{lem:PathCorr}, there is a path $x_1, x_2, \ldots, x_k$ in $\Gamma$, with $x_i \in X_i$, $i = 1, 2, \ldots, k$. Now, since $X_k \adj{\Gamma'} X_1$, Lemma~\ref{lem:xy=>x'y'} implies that $x_k \adj{\Gamma} y_1$, for some vertex $y_1 \in X_1$. If $y_1 = x_1$, we obtain a cycle as required. If not, we proceed as before to find a path $y_1, y_2, \ldots, y_k$ with $y_i \in X_i$, $i = 1, 2, \ldots, k$. As there are finitely many vertices, this procedure terminates at a repeated vertex, and we obtain a cycle of the required form in $\Gamma$.
\end{proof}

\end{document}